\documentclass[10pt, a4paper]{article}
    \usepackage[margin=0.9cm]{geometry}
    \usepackage{array}
    \usepackage{xcolor}
    \usepackage{fontawesome}
    \usepackage{enumitem}
    \usepackage{hyperref}
    \usepackage{titlesec}
    \pagenumbering{gobble}
    \titlespacing*{\section}{0pt}{0.225\baselineskip}{0.225\baselineskip}
    \setlist[itemize]{noitemsep, topsep=0pt}
    \newcolumntype{L}{>{\raggedleft}p{0.11\textwidth}}
    \newcolumntype{R}{p{0.84\textwidth}}
    \setlength{\parindent}{0pt}
    \newcommand\vsep{\color{lightgray} \vrule width 0.5pt}
    \newcommand\sect[1]{\section*{\Large\sc #1}}
    \begin{document}
        \begin{center}
            \bfseries\huge\sc Eugene Lin
        \end{center}
        \
        \begin{tabular*}{0.99\textwidth}{@{\extracolsep{\fill}} ccccc}
            \faPhone \ \ +44 7577 490072 &
            \faEnvelope \ \ \href{mailto:me@eugenel.in}{me@eugenel.in} &
            \faGithub \ \ \href{https://github.com/lin-e/}{github.com/lin-e} &
            \faGlobe \ \ \href{https://eugenel.in/}{eugenel.in} &
            \faLinkedinSquare \ \ \href{https://www.linkedin.com/in/line/}{linkedin.com/in/line}
        \end{tabular*}
        \sect{Education}
            \begin{tabular}{L!{\vsep}R}
                2018 - 2022 & MEng Computing, Imperial College London
                \begin{itemize}
                    \item Year 1: 81\% average, with first-class honours in all modules, and 85\% in programming
                    \vspace{-\baselineskip}
                \end{itemize} \\
                2011 - 2018 & Dunraven School (Sixth Form, and Secondary School)
                \begin{itemize}
                    \item STEP: Grade 2 in STEP I, and Grade 3 in STEP II
                    \item A Level: A*A*A*A in Maths, Further Maths, Computer Science, and Physics
                    \item BMO Round 1: Perfect score in 2 questions; qualified based on UKMT SMC results
                    \item GCSE: A\^{} (Further Maths), 8 A*s (equivalent), and 5 As
                    \vspace{-\baselineskip}
                \end{itemize}
            \end{tabular}
        \sect{Work Experience}
            \begin{tabular}{L!{\vsep}R}
                Fire Tech Camp & \textbf{(2019)}
                    Worked as a Senior Java/C\# instructor, for groups of children aged 14, and up.
                    I received consistent positive feedback; especially due to my hands-on teaching style.
                    This position provided me with experience on debugging code written by others.
                    \\
                Tutoring & \textbf{(2016 - 2019)}
                    For a period of time, I tutored children from Year 6 to Year 10, with a focus in mathematics.
                    During this time, I was able to improve my communication skills, and practised how to explain challenging topics in simple terms.
                    \\
                Freelance Projects & \textbf{(2015 - 2018)}
                    I occasionally take on clients for freelance work.
                    In all of my projects, I aim to meet the expectations of my clients via a system of consistent communication and updates on the status of the project, as well as taking any inputs or suggestions they might have.
                    \\
                Zodiac Media & \textbf{(2017)}
                    During my work experience placement, I assisted in updating a legacy e-commerce site running Magento.
                    This helped to develop my skills using version control, alongside other tools; such as setting up Vagrant-based VMs for web development.
            \end{tabular}
        \sect{Projects}
            \begin{tabular}{L!{\vsep}R}
                C Group Project & \textbf{(2019)}
                    Part of our first year C module.
                    Implemented an assembler, and emulator for the ARMv6 instruction set.
                    Designed custom hardware (PCB), as well as the software interface for a bi-colour LED matrix.
                    Also assisted in implementing the extension - a Monte-Carlo Tree Search to play Connect 4.
                    \\
                eduCATe & \textbf{(2019)}
                    For ICHack 19, our team chose to implement a chat platform, being tailored for educational purposes.
                    Our design was to use a similar threading system to Slack for each assignment, in order to maintain organisation.
                    I designed, and implemented, the backend (built mostly on PHP), with a focus on cross-platform compatibility.
                    The implementation we decided to go was with an Android application, as it was something we haven't tried before.
                    \\
                Yeetr & \textbf{(2018)}
                    During HackKing's 5.0, we decided to recreate Twitter from scratch (for a 90's themed challenge).
                    Other than jQuery, and Google's OAuth2, we used no other external libraries.
                    Through this project, we learnt how to deploy through Git, set up a basic LAMP stack, and prevent against basic attacks.
                    While I mainly worked on the endpoints, I assisted in implementing some JavaScript to handle features such as a refreshing feed.
                    \\
                Untitled Game & \textbf{(2018)}
                    As part of my final project for A Level Computer Science, I wrote a game in C\#, with Unity.
                    The project featured procedural terrain generation, which created mazes with basic MST algorithms, and generated trees with a custom mesh builder.
                    The entities in the game gradually improved with the use of a genetic algorithm, and also kept track of scores with a basic web interface.
                    The game also implements a low-latency LAN library I designed for interacting with IoT lighting devices.
                    \\
                Skype4Sharp & \textbf{(2016 - 2017)}
                    The library is outdated, and I haven't had time to fix it due to other commitments.
                    At the time, it was one of the only (if not the only) .NET libraries for Skype that allowed for interaction with the web client, and improved on aspects of the original Skype4COM framework.
                    This is unlikely to be updated, due to Skype's decreasing popularity.
                    Also designed an open-sourced Skype bot utilising this library, featuring a simple plugin manager, alongside a basic permissions system.
            \end{tabular}
        \sect{Skills \& Interests}
            \begin{tabular}{L!{\vsep}R}
                Programming &
                    Despite the majority of my experience being in C\# (including Unity), PHP, Python, and Java; I have various amounts of experience in other programming languages, including Haskell, Kotlin, C, and JavaScript (mainly with the p5.js framework).
                    All of my public projects are available on my GitHub.
                    I am also comfortable with markup languages such as LaTeX.
                    \\
                Linux &
                    Proficient in using the Linux command line - including managing headless servers.
                    Experience working with web stacks such as LAMP, and MEAN.
                    Able to use standard programming utilities such as version control.
                    Comfortable with automating (and scripting) repetitive tasks.
                    \\
                Mandarin &
                    I am fluent in spoken Mandarin Chinese, and am able to converse at a reasonable pace with other native speakers.
                    \\
                Prototyping \& Design &
                    Due to an interest in customising devices (namely keyboards), I started to explore different parts of prototyping, such as designing my own PCBs.
                    As such, I completed an additional course in prototyping at Imperial College London, which aided in developing my skills with electronics, as well as teaching me the basics of CAD. I also have some experience in Photoshop, and After Effects.
            \end{tabular}
    \end{document}