\documentclass[10pt, a4paper]{article}
    \usepackage[margin=1cm]{geometry}
    \usepackage{array}
    \usepackage{xcolor}
    \usepackage{fontawesome}
    \usepackage{enumitem}
    \usepackage{hyperref}
    \usepackage{titlesec}
    \pagenumbering{gobble}
    \titlespacing*{\section}{0pt}{0.5\baselineskip}{0.4\baselineskip}
    \setlist[itemize]{noitemsep, topsep=0pt}
    \newcolumntype{L}{>{\raggedleft}p{0.12\textwidth}}
    \newcolumntype{R}{p{0.8\textwidth}}
    \setlength{\parindent}{0pt}
    \newcommand\vsep{\color{lightgray} \vrule width 0.5pt}
    \begin{document}
        \begin{center}
            \bfseries\huge\sc Eugene Lin
        \end{center}
        \begin{center}
            \faPhone \ \ +44 7577 490072 \ \ \ \ \ \ \ \ \
            \faEnvelope \ \ \href{mailto:me@eugenel.in}{me@eugenel.in} \ \ \ \ \ \ \ \ \
            \faGithub \ \ \href{https://github.com/lin-e/}{github.com/lin-e} \ \ \ \ \ \ \ \ \
            \faGlobe \ \ \href{https://eugenel.in/}{eugenel.in}
        \end{center}
        \section*{\sc Education}
            \begin{tabular}{L!{\vsep}R}
                2018 - 2022 & MEng Computing, Imperial College London \\
                2011 - 2018 & Dunraven School (Sixth Form, and Secondary School)
                \begin{itemize}
                    \item A Level: 3 A*s in Maths, Further Maths, Computer Science, and A in Physics
                    \item GCSE: A\^{} (Further Maths), 8 A*s (equivalent), and 5 As
                    \vspace{-\baselineskip}
                \end{itemize}
            \end{tabular}
        \section*{\sc Work Experience}
            \begin{tabular}{L!{\vsep}R}
                Tutoring & \textbf{(2016 - 2019)} For a period of time, I tutored children from Year 6 to Year 10, with a focus in mathematics. During this time, I was able to improve my communication skills, and practiced how to explain challenging topics in simple terms. \\
                Freelance Projects & \textbf{(2015 - 2018)} During my free time, I occasionally take on clients for freelance work. In all of my projects, I aim to meet the expectations of my clients via a system of consistent communication and updates on the status of the project, as well as taking any inputs or suggestions they might have. \\
                Zodiac Media & \textbf{(2017)} During my work experience placement, I assisted in updating a legacy site running Magento, as well as developed my skills using Git, alongside other tools; such as setting up Vagrant-based VMs for web development.
            \end{tabular}
        \section*{\sc Projects}
            \begin{tabular}{L!{\vsep}R}
                eduCATe & \textbf{(2019)} For ICHack 19, our team chose to implement a chat platform, being tailored for educational purposes. We noticed that we'd often lose important information in a huge chat, but existing forums seem too formal. Our design was to use a similar threading system to Slack for each assignment, in order to maintain organization. I designed, and implemented, the backend (built mostly on PHP), with a focus on compatibility for other platforms. The implementation we decided to go was with an Android application, as it was something we haven't tried before. \\
                Yeetr & \textbf{(2018)} During HackKing's 5.0, we decided to recreate Twitter from scratch (for a 90's themed challenge). Other than jQuery, and Google's OAuth2, we used no other external libraries. Through this project, we learnt how to deploy through Git, set up a basic LAMP stack, and prevent against basic attacks. While I mainly worked on the backend, I assisted in implementing some JavaScript to handle features such as a refreshing feed. \\
                Untitled Game & \textbf{(2018)} As part of my final project for A Level Computer Science, I wrote a game in C\#, with Unity. The project featured procedural terrain generation, which created mazes with basic MST algorithms, and generated trees with a custom mesh builder. The entities in the game gradually improved with the use of a genetic algorithm, and also kept track of scores with a basic web interface. \\
                Skype4Sharp & \textbf{(2016)} The library is outdated, and I haven't had time to fix it due to other commitments. At the time, it was one of the only (if not the only) .NET libraries for Skype that allowed for interaction with the web client, and improved on aspects of the original Skype4COM framework. This is unlikely to be updated, due to Skype's decreasing popularity. \\
                SimpleSkype & \textbf{(2016)} Originally created to replace another Skype chatbot available at the time, since it was utilizing an outdated library. The project is open-sourced, and features a simple plugin manager, alongside a basic permissions system. \\
            \end{tabular}
        \section*{\sc Skills, Achievements \& Interests}
            \begin{tabular}{L!{\vsep}R}
                Programming & Despite the majority of my experience being in C\# (including Unity), PHP, and Python; I've had varied amounts of experience in other programming languages, including Haskell, Java, and JavaScript (mainly with the p5.js framework). All of my public projects are available on my GitHub. \\
                Mandarin & I am fluent in spoken Mandarin Chinese, and am able to converse at a reasonable pace with other native speakers. \\
                BMO 2017 & I qualified for the first round of the British Maths Olympiad based on my score in the UKMT Senior Maths Challenge, with a perfect score in two BMO questions. I've also competed in a variety of UKMT challenges, both individually, and as part of a team. \\
                Prototyping & Due to an interest in customizing devices (namely keyboards), I started to explore different parts of prototyping and design. As such, I'm currently taking an additional course in product design at Imperial College London, which helps to develop my skills with electronics, as well as teaching me the basics of CAD (and rapid prototyping).
            \end{tabular}
    \end{document}